% -*- coding=utf-8 -*-
\chapter[基础]{基础}  % \chapter[短标题]{长标题} 页眉会显示短标题,正文会显示长标题,“\\"会强制换行,”\ “会显示空格题目:要体现专业特点,应精练、准确地概括论文研究的主要内容和结果,一般不超过 20 字。
\songti\zihao{-4}
\linespread{1.67} \selectfont

本模板为西北大学学士学位论文的 \LaTeX 排版示例,参考自\href{https://www.overleaf.com/latex/templates/shan-xi-shi-fan-da-xue-li-gong-ke-lei-ben-ke-sheng-bi-ye-lun-wen-texmo-ban/rkxzpttqqzyt}{\color{red}{陕西师范大学本科生毕业论文模板}},并依据学校提供的论文写作规范进行了适当修改与优化。如模板中存在不符合个人或学校实际要求之处,可根据需要自行调整。

\section{文档说明}

\subsection{准备工作}

在 \textbf{\textcolor{green}{Overleaf}} 平台上使用本模板时,只需在菜单栏中将编译器设置为 \textbf{\textcolor{green}{XeLaTeX}},即可正常编译运行。

若在本地使用,需先安装 \textbf{\textcolor{green}{TeX Live 2024}} 作为编译环境,并选择合适的编辑器(如 \textbf{\textcolor{green}{TeXstudio}} 或 \textbf{\textcolor{green}{Visual Studio Code}})。安装方法可参考互联网资源(如 \textbf{\textcolor{green}{Bilibili}}、\textbf{\textcolor{green}{CSDN}}、\textbf{\textcolor{green}{博客园}}等),或选择在 \textbf{\textcolor{green}{淘宝}}、\textbf{\textcolor{green}{京东}} 等平台寻求远程安装服务。

\subsection{编译方法}

在 \textbf{\textcolor{green}{Overleaf}} 中,点击“编译”按钮即可完成编译。

在本地使用 \textbf{\textcolor{green}{VS Code}} 时,若仅需快速编译正文而不更新参考文献,可直接选择 \textbf{\textcolor{green}{XeLaTeX}} 编译,以加快速度;若需要重新生成参考文献,则应选择 \textbf{\textcolor{green}{XeLaTeX \& Biber}} 编译方式。


\section{模板文件结构}

本节介绍模板的文件结构,该模板采用配置与内容分离的设计,主要包含根目录下的配置文件main.tex以及三个子目录bib,figure,tex。

如果想要修改全局的配置,就去main.tex;
想要编辑论文的内容,就去tex/目录;
图片都放到figure/目录;
参考文献数据放在bib/目录。

\subsection{配置文件main.tex}

配置文件main.tex的作用在于定义全局配置,例如文档类型,引入宏包,页面布局等,可以理解为tex文件的导言区。

\subsection{内容目录tex/}

tex目录下共有九个文件tex文件,分别对应于封面,原创性声明页,中文摘要,英文摘要,正文页,总结页,参考文献,致谢,研究成果页。此外还有一个子目录chapters/用来存放正文页导入的内容。

封面页只需要修改最开始的几个参数即可。

原创性声明页一般无需改动。

中文摘要,英文摘要,正文页,总结页,致谢需要自行编辑。

研究成果页已经给出了示例,仿造一下即可。

\subsection{参考文献目录bib/}

目录下有一个参考文献数据库文件\verb|ref.bib|来存放学位论文参考的文献信息。

\subsection{图片目录figure/}

存放需要用到的图片,本模板可以使用的格式包括pdf,jpg,png,eps。
配置文件中已经定义了图片的存放路径,所以插入图片的时候,顶层目录figure/可以省略。作者建议不要省略,可读性更强一些。
