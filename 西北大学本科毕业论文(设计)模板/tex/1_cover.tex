% -*- coding=utf-8 -*-   % 指定文件编码为 UTF-8,确保中文正常显示

%\newcommand\clcNumber{O175.3}    % 分类号(可选),如不需要可注释掉
                                  % 二级学科名称(此处为空,可自行添加)
\newcommand\addtion{教务处制}     % 附加说明,通常为制表单位
\newcommand{\thesisTitle}{XXXX}   % 论文标题
\newcommand{\thesisAuthor}{XX}    % 学生姓名
\newcommand{\studentId}{xxxxxxxx} % 学号
\newcommand{\supervisor}{xxxxx}   % 指导教师姓名
\newcommand{\institute}{xx学院}   % 学院名称
\newcommand{\major}{xxxxxx}       % 专业名称
\newcommand{\grade}{xxxxx}        % 年级

% ============================= 封面页面设置(如有必要可自行更改) =============================

% 封面背景为已导出的 PDF 文件(建议封面背景图为无白边且对齐的 PDF)
{
    \setlength\parindent{0em} % 取消首行缩进
    \ThisTileWallPaper{\paperwidth}{\paperheight}{figure/nwucover.pdf} % 设置封面背景图片
    \vspace*{0.2cm} % 设置页面内容的上边距

    \vspace*{9.5cm} % 控制标题区域的垂直位置

    % ============================= 论文标题设置 =============================
    % 如果标题较短,可使用以下单行写法(已注释):
    % \begin{center}
    %     \zihao{3}  题目:\underline{\makebox[9cm][c]{\heiti\zihao{2}\thesisTitle}}
    % \end{center}

    % 如果标题较长,可使用如下分两行写法(当前启用):
    \begin{center}
        {\zihao{3}  题目:}   \underline{\makebox[22em][c]{\zihao{2}\heiti 我的本科毕业论文题目}} \par
        {\zihao{-3} \qquad\quad }   \underline{\makebox[22em][c]{\zihao{2}\heiti 实在是太长了 }} \par
    \end{center}

    \vspace{0.4cm} % 控制标题与学生信息之间的垂直间距

    \renewcommand{\baselinestretch}{2.25}\selectfont % 设置行间距

    % ============================= 学生信息区域 =============================
    {
        \begin{center}
        \songti % 设置字体为宋体
        {\zihao{-3} 学生姓名} \underline{\makebox[10em][c]{\zihao{-3}\thesisAuthor}} \par
        {\zihao{-3} 学\;\;\;\;\;\;\;号} \underline{\makebox[10em][c]{\zihao{-3}\studentId}} \par
        {\zihao{-3} 指导教师} \underline{\makebox[10em][c]{\zihao{-3}\supervisor}} \par
        {\zihao{-3} 院\;\;\;\;\;\;\;系} \underline{\makebox[10em][c]{\zihao{-3}\institute}} \par
        {\zihao{-3} 专\;\;\;\;\;\;\;业} \underline{\makebox[10em][c]{\zihao{-3}\major}} \par
        {\zihao{-3} 年\;\;\;\;\;\;\;级} \underline{\makebox[10em][c]{\zihao{-3}\grade}} \par
        \end{center}
    }
}
