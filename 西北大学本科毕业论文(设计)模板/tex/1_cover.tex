% -*- coding=utf-8 -*-
%\newcommand\clcNumber{O175.3}                        % 分类号
                     % 二级学科名称
\newcommand\addtion{教务处制} 
\newcommand{\thesisTitle}{XXXX}
\newcommand{\thesisAuthor}{XX}
\newcommand{\studentId}{xxxxxxxx}
\newcommand{\supervisor}{xxxxx}
\newcommand{\institute}{xx学院}
\newcommand{\major}{xxxxxx}
\newcommand{\grade}{xxxxx}


%============================================如有必要,以下内容可自行更改
%封面背景为已经导出的具有偏移的PDF封面
{                                                     
    \setlength\parindent{0em}                                               % 设置首行缩进为0
    \ThisTileWallPaper{\paperwidth}{\paperheight}{figure/nwucover.pdf}         % 设置封面背景
    \vspace*{0.2cm}                                                         % 设置页面内容上间距
 
    \vspace*{9.5cm}
   % \begin{center}
        %\zihao{3}  
       % 题目:\underline{\makebox[9cm][c]{\heiti\zihao{2}\thesisTitle}}
    %\end{center}
%% 如果你的标题太长,想分两行,可以把上面这四行代码换成下面这段注释了的代码。
     \begin{center}
         {\zihao{3}  题目:}   \underline{\makebox[22em][c]{\zihao{2}\heiti 我的本科毕业论文题目}} \par
        {\zihao{-3} \qquad\quad }   \underline{\makebox[22em][c]{\zihao{2}\heiti 实在是太长了 }} \par
    \end{center}
    \vspace{0.4cm}
    \renewcommand{\baselinestretch}{2.25}\selectfont                           % 行间距
    {
        \begin{center}
        \songti % 设置字体为宋体
        {\zihao{-3} 学生姓名} \underline{\makebox[10em][c]{\zihao{-3}\thesisAuthor}} \par
        {\zihao{-3} 学\;\;\;\;\;\;\;号} \underline{\makebox[10em][c]{\zihao{-3}\studentId}} \par
        {\zihao{-3} 指导教师} \underline{\makebox[10em][c]{\zihao{-3}\supervisor}} \par
        {\zihao{-3} 院\;\;\;\;\;\;\;系} \underline{\makebox[10em][c]{\zihao{-3}\institute}} \par
        {\zihao{-3} 专\;\;\;\;\;\;\;业} \underline{\makebox[10em][c]{\zihao{-3}\major}} \par
        {\zihao{-3} 年\;\;\;\;\;\;\;级} \underline{\makebox[10em][c]{\zihao{-3}\grade}} \par
\end{center}
    }
}